\chapter[CONCLUSÕES]{CONCLUSÕES}

Este trabalho abordou uma comparação entre diversos algoritmos para suavização de malhas não estruturadas e o uso do algoritmo genético como um novo processo adaptativo. Foi apresentado o fundamento teórico que mostra a importância da geração e suavização de malhas para a solução numérica em malhas não estruturadas.

O algoritmo criado para comparar os resultados usa o paradigma de orientação a objetos, o que permite uma melhor organização do código computacional. As classes e suas estruturas são apresentadas no capítulo 9.

O algoritmo genético melhorou a solução inicial, no entanto, teve um desempenho inferior tanto nos resultados gerados quanto no custo computacional em relação a outros métodos conhecidos.